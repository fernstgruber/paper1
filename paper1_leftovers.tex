%\documentclass[preprint,12pt,authoryear]{elsarticle}
\documentclass[final,1p,times,twocolumn,authoryear]{elsarticle}
\usepackage{lineno,hyperref}
\modulolinenumbers[5]

\journal{Journal of \LaTeX\ Templates}

%%%%%%%%%%%%%%%%%%%%%%%
%% Elsevier bibliography styles
%%%%%%%%%%%%%%%%%%%%%%%
%% To change the style, put a % in front of the second line of the current style and
%% remove the % from the second line of the style you would like to use.
%%%%%%%%%%%%%%%%%%%%%%%

%% Numbered
%\bibliographystyle{model1-num-names}

%% Numbered without titles
%\bibliographystyle{model1a-num-names}

%% Harvard
\bibliographystyle{model2-names.bst}\biboptions{authoryear}

%% Vancouver numbered
%\usepackage{numcompress}\bibliographystyle{model3-num-names}

%% Vancouver name/year
%\usepackage{numcompress}\bibliographystyle{model4-names}\biboptions{authoryear}

%% APA style
%\bibliographystyle{model5-names}\biboptions{authoryear}

%% AMA style
%\usepackage{numcompress}\bibliographystyle{model6-num-names}

%% `Elsevier LaTeX' style
%\bibliographystyle{elsarticle-harv}
%%%%%%%%%%%%%%%%%%%%%%%

\begin{document}

\begin{frontmatter}

\title{LEFTOVERS}


%% Group authors per affiliation:
\author{Fabian E. Gruber\fnref{myfootnote}}
\author{Jasmin Baruck\fnref{myfootnote}}
\author{Clemens Geitner\fnref{myfootnote}}

\address{University of Innsbruck}

\begin{abstract}
These are the leftovers from the removal of GL1 in my paper ``algorithms vs. surveyors''
\end{abstract}

\begin{keyword}

\end{keyword}

\end{frontmatter}

\linenumbers

\section{Introduction}

\section{Material and Methods}
For generalisation level (GL) 2, data points from classes with very few members were left out or included into different classes. 

\begin{table}[!htbp]
\caption{The topographic positions as mapped by the surveyor were reclassified to investigate whether different classifications improve the investigation of the differentiating parameters. FL = flat, LO = hydrologically low area, DA = debris accumulation; See Table \ref{table:topopositions} for further abbreviations. }
\begin{center}
    \begin{tabular}{  p{2.5cm} c  c  c  c }
	\hline\hline
	Topographic position & macro gen. 1 & macro gen. 2 & meso gen. 1 & meso gen. 2 \\ \hline
	EF & FL & - & FL & -\\ 
	LF & FL & - & FL & -  \\ 
	VF & FL & FS & FL & FS \\ 
	TE & FL & - & FL & - \\ 
	PL & FL & SH & FL & SH \\ 
	DE & LO & - & LO & - \\ 
	PA & LO & - & LO & - \\ 
	CH & LO & CH & LO & - \\ 
	SH & SH & SH & SH & SH \\ 
	FS & FS & FS & FS & FS \\ 
	BS & BS & BS & BS & BS \\ 
	SS & - & - & SS & - \\ 
	SF & - & - & SF & - \\ 
 	SU & RI & SH & RI & SH \\ 
	RI & RI & SH & RI & SH \\ 
	RA & RI & - & RI &	- \\ 
	TS & FS & FS & FS & FS \\ 
	DF & DA & FS & DA & FS \\ 
	DC & DA & FS & DA & FS \\
	GR & - & - & - & - \\
	SV & - & - & - & - \\ 
    \end{tabular}
    \label{table:generalisations}
\end{center}
\end{table}
 
\subsection{Terrain parameters and roughness}
\citep{Riley1999}
\section{Results}
\subsection{Dikau}
\begin{table}[!htbp]
\caption{The topographic positions as mapped by the surveyor were reclassified to investigate whether different classifications improve the investigation of the differentiating parameters. FL = flat, LO = hydrologically low area, DA = debris accumulation; See Table \ref{table:topopositions} for further abbreviations. }
\begin{center}
    \begin{tabular}{  p{2.5cm} c  c  }
	\hline\hline
	Topographic position & macro gen.  & meso gen.  \\ \hline
	EF & FL  & FL \\ 
	LF & FL  & FL  \\ 
	VF & FL  & FL  \\ 
	TE & FL  & FL  \\ 
	PL & FL  & FL  \\ 
	DE & LO  & LO  \\ 
	PA & LO  & LO \\ 
	CH & LO  & LO  \\ 
	SH & SH  & SH  \\ 
	FS & FS  & FS  \\ 
	BS & BS  & BS \\ 
	SS &  -  & SS  \\ 
	SF &  -  & SF  \\ 
 	SU & RI  & RI \\ 
	RI & RI  & RI  \\ 
	RA & RI  & RI  \\ 
	TS & FS  & FS  \\ 
	DF & DA  & DA  \\ 
	DC & DA  & DA \\
	GR &  -  &  - \\
	SV &  -  &  - \\ 
    \end{tabular}
    \label{table:generalisations2}
\end{center}
\end{table}
\begin{table}[!htbp]
\caption{Accuracy values of  Dikau's curvature classification maps computed  with the best parameter setting for each topographic position at macro scale. "plus"  indicates that the parameter setting described is combined with the aforementioned parameter setting. res = DTM resolution,tc = curvature threshold for plane, Ov = Overall accuracy}
\centering
\begin{tabular}{p{3.2cm}|rrrrrrrr}
  \hline
setting at GL 1 & FL & LO & DA & FS &  BS  & SH & RI & Ov \\ 
  \hline
res=100m tc=0.006  & \raisebox{-0ex}{0.00} & \raisebox{-0ex}{0.49} & \raisebox{-0ex}{0.20} & \raisebox{-0ex}{0.00} & \raisebox{-0ex}{0.81} & \raisebox{-0ex}{0.00} & \raisebox{-0ex}{0.37} & \raisebox{-0ex}{0.49}\\ 
 \hline
 setting at GL 2 & CH &  &  & FS &  BS  & SH &  & Ov \\ 
  \hline
\raisebox{-1.5ex}{res=100m tc=0.006} & \raisebox{-1.5ex}{0.28}  &  &  & \raisebox{-1.5ex}{0.14} &   \raisebox{-1.5ex}{0.87} & \raisebox{-1.5ex}{0.22} &  & \raisebox{-1.5ex}{0.47} \\ 
\raisebox{-0.5ex}{plus} & \raisebox{-1.5ex}{0.33}  &  &  & \raisebox{-1.5ex}{0.09} &   \raisebox{-1.5ex}{0.78} & \raisebox{-1.5ex}{0.39} &  & \raisebox{-1.5ex}{0.49} \\ 
\raisebox{0.5ex}{res=250m tc=0.0016}  &  &  &  &  &    &  &  &\\
  \hline
\end{tabular}
\label{table:dikau_macro}
\end{table}
The dominance of BS persists at generalisation level 2, where this parameter setting leads to a cross-validated accuracy of 47\% due to the reduction of topographic position classes. 

macro:
However, at GL2 the addition of a map based on the coarsest DTM at grid cell size  250 m and a plane threshold of 0.0016 enhances the overall correct classification rate to 49\% due to increased classification of CH and SH at the expense of the accuracy of FS and BS.
 Table \ref{table:dikau_macro} gives a summary of the results at macro scale.
 
meso:
 At GL 2  a smaller plane threshold of 0.003 was chosen as the best parameter setting. This however results in a map which only maps BS and SH as topographic positions, with X/X being the only curvature class not mapped to BS. While the addition of the map based on the 100 m DTM and a plane threshold of 0.01 increases the overall accuracy by only 1\% to 61\%, it does lead to the inclusion of the class FS into the resulting map of topographic positions.
\begin{table}[!htbp]
\caption{Accuracy values of  Dikau's curvature classification maps computed  with the best parameter setting for each topographic position at meso scale. "plus"  indicates that the parameter setting described is combined with the aforementioned parameter setting. res = DTM resolution,tc = curvature threshold for plane, Ov = Overall accuracy}
\centering
\begin{tabular}{p{3cm}|rrrrrrrrrr}
  \hline
Setting at GL 1 & FL & LO & DA & FS & SF & BS & SS & SH & RI & Ov \\ 
  \hline

{res=50m tc=0.007} & {0.00} & {0.26} &{0.00} & {0.00} & {0.00} & {0.93} & {0.00} & {0.0} & {0.21} & {0.47} \\ 
plus res=100m tc=0.006 & \raisebox{-1.5ex}{0.00} & \raisebox{-1.5ex}{0.24} & \raisebox{-1.5ex}{0.00} & \raisebox{-1.5ex}{0.01} & \raisebox{-1.5ex}{0.00} & \raisebox{-1.5ex}{0.91} & \raisebox{-1.5ex}{0.00} & \raisebox{-1.5ex}{0.10} & \raisebox{-1.5ex}{0.33} & \raisebox{-1.5ex}{0.49} \\
 \hline
 Setting at GL 2 &  &  &  & FS &  & BS & & SH &  & Ov \\ 
 \hline
\raisebox{-0ex}{res=50m tc=0.003} &  &  &  & \raisebox{-0ex}{0.00} &  & \raisebox{-0ex}{0.86} & & \raisebox{-0ex}{0.42} &  & \raisebox{-0ex}{0.60} \\ 
plus res=50m tc=0.01&  &  &  & \raisebox{-1.5ex}{0.15} &  & \raisebox{-1.5ex}{0.84} & & \raisebox{-1.5ex}{0.44} &  & \raisebox{-1.5ex}{0.61} \\ 
  \hline
\end{tabular}
\label{table:dikau_meso}
\end{table} 
 \subsection{Wood}
 Macro:
At GL 2, the single map with the best correct classification rate of 48\% is the same as at GL1, classifying the points into 3 out of 4 possible  topographic positions while leaving out FS. Introducing an additional map based on a larger window size (1250 m) and a smaller curvature threshold (0.007) adds the topographic position FS to the output, increasing overall accuracy to 51\%. 
 The addition of a different parameter setting with a window size of 150 m, slope threshold of 8 degrees and a smaller curvature threshold of 0.00001, results in a minor increase of the overall accuracy to 47\%, but improves the number of topographic positions mapped by introducing the position FL.
 
 Meso:
 The reduction of the number of topographic positions to 3 at GL 2 results in a correct classification rate of 59\%, with the 10-fold cross validation not resulting in a clear victory of a specific parameter setting. The forerunners perform well at predicting the dominant class BS, and have problems predicting the two other classes, especially FS.
 
  
 \begin{table}[!htbp]
\caption{Accuracy values of  Wood's parametric feature maps computed  with the best parameter setting for each topographic position at macro scale. "plus"  indicates that the parameter setting described is combined with the aforementioned parameter setting. res = grid cell size, ws = window size, ts = slope threshold tc = curvature threshold, Ov = Overall accuracy}
\centering
\begin{tabular}{p{2.8cm}|rrrrrrrr}
  \hline
setting at GL 1 & FL & LO & DA & FS &  BS  & SH & RI & OCR \\ 
  \hline
res=50m ws=250m ts=14 tc=0.002 & \raisebox{-1.5ex}{0.00} & \raisebox{-1.5ex}{0.39} & \raisebox{-1.5ex}{0.00} & \raisebox{-1.5ex}{0.00} & \raisebox{-1.5ex}{0.80} & \raisebox{-1.5ex}{0.00} & \raisebox{-1.5ex}{0.36} & \raisebox{-1.5ex}{0.46}  \\ 
 \hline
 setting at GL 2 & CH &  &  & FS &  BS  & SH &  & OCR \\ 
  \hline
res=50m ws=250m ts=14 tc=0.002 & \raisebox{-1.5ex}{0.4}  &  &  & \raisebox{-1.5ex}{0.00} &   \raisebox{-1.5ex}{0.80} & \raisebox{-1.5ex}{0.34} &  & \raisebox{-1.5ex}{0.48} \\ 
plus res=50m ws=1250m ts=14 tc=0.007 & \raisebox{-1.5ex}{0.4}  &  &  & \raisebox{-1.5ex}{0.19} &   \raisebox{-1.5ex}{0.74} & \raisebox{-1.5ex}{0.41} &  & \raisebox{-1.5ex}{0.51} \\ 
  \hline
\end{tabular}
\label{table:wood_macro}
\end{table}


\begin{table}[!htbp]
\caption{Accuracy values of  Wood's parametric feature maps computed  with the best parameter setting for each topographic position at meso scale. "plus"  indicates that the parameter setting described is combined with the aforementioned parameter setting. res = grid cell size, ws = window size, ts = slope threshold tc = curvature threshold, OCR = Overall correct classification rate}
\centering
\begin{tabular}{p{2.8cm}|rrrrrrrrrr}
  \hline
 & FL & LO & DA & FS & SF & BS & SS & SH & RI & OCR \\ 
  \hline
res=10m ws=110m ts=11 tc=0.006 & \raisebox{-1.5ex}{0.00} & \raisebox{-1.5ex}{0.39} & \raisebox{-1.5ex}{0.00} & \raisebox{-1.5ex}{0.00} & \raisebox{-1.5ex}{0.00} & \raisebox{-1.5ex}{0.90} & \raisebox{-1.5ex}{0.00} & \raisebox{-1.5ex}{0.00} & \raisebox{-1.5ex}{0.25} & \raisebox{-1.5ex}{0.47} \\ 
 \hline
  Setting at GL 2 &  &  &  & FS &  & BS & & SH &  & OCR \\ 
  \hline
 res=10m ws=150m ts=15 tc=0.004 &  &  &  & \raisebox{-1.5ex}{0.17} &  & \raisebox{-1.5ex}{0.88} & & \raisebox{-1.5ex}{0.29} &  & \raisebox{-1.5ex}{0.59} \\ 
  \hline
\end{tabular}
\label{table:wood_meso}
\end{table}

\subsection{Fuzzy}

macro:

 The addition of a second predictor map based on the 50 m DTM, slope thresholds of 3 and 12, and curvature thresholds of 0.0008 and 0.006 returns the same overall classification rate but increases the number of correctly classified RI positions at the cost of correctly identied BS positions. At GL 2, the parameter setting that performed best, with a cross-validated accuracy of 53\%, is based on the 150 m DTM as well, but with a wider range of slope thresholds (3° and 21°) and a smaller range of curvature thresholds(0.001 and 0.0014). Superposition with a map based on a different parameter setting leads to no improvisation of overall accuracy nor the correct classification rate of individual classes.
\begin{table}[!htbp]
\caption{Accuracy values of  Schmidt's fuzzy element maps computed  with the best parameter setting for each topographic position at macro scale. "plus"  indicates that the parameter setting described is combined with the aforementioned parameter setting. res = DTM resolution, ts = slope thresholds, tc = curvature thresholds, Ov = Overall accuracy}
\centering
\begin{tabular}{p{2.8cm}|rrrrrrrr}
  \hline
setting at GL 1 & FL & LO & DA & FS &  BS  & SH & RI & Ov \\ 
  \hline
res=150m ts=6;12 tc=0.0001;0.003 & \raisebox{-1.5ex}{0.38} & \raisebox{-1.5ex}{0.36} & \raisebox{-1.5ex}{0.00} & \raisebox{-1.5ex}{0.32} & \raisebox{-1.5ex}{0.81} & \raisebox{-1.5ex}{0.00} & \raisebox{-1.5ex}{0.29} & \raisebox{-1.5ex}{0.49}  \\ 
 \hline
 setting at GL 2 & CH &  &  & FS &  BS  & SH &  & Ov \\ 
  \hline
res=150m ts=3;21 tc=0.001;0.0014 & \raisebox{-1.5ex}{0.42}  &  &  & \raisebox{-1.5ex}{0.34} &   \raisebox{-1.5ex}{0.70} & \raisebox{-1.5ex}{0.44} &  & \raisebox{-1.5ex}{0.53} \\ 
  \hline
\end{tabular}
\label{table:fuzzy_macro}
\end{table}

meso:

At GL 2 the result of the cross validation process is similarily indecisive, with an accuracy of approximately 63\%. While the range of the slope thresholds is decreased due to a lower threshold value of 15$^{\circ}$, both curvature thresholds are smaller than at GL 1, with 0.0002 and 0.002 as lower and upper boundaries, respectively. As with GL 1,


\begin{table}[!htbp]
\caption{Accuracy values of  Schmidt's fuzzy element maps computed  with the best parameter setting for each topographic position at meso scale. "plus"  indicates that the parameter setting described is combined with the aforementioned parameter setting. res = DTM resolution, ts = slope thresholds, tc = curvature thresholds, Ov = Overall accuracy}
\centering
\begin{tabular}{p{2.8cm}|rrrrrrrrrr}
  \hline
 & FL & LO & DA & FS & SF & BS & SS & SH & RI & Ov \\ 
  \hline
res=50m ts=3;21 tc=0.004;0.006 & \raisebox{-1.5ex}{0.00} & \raisebox{-1.5ex}{0.48} & \raisebox{-1.5ex}{0.00} & \raisebox{-1.5ex}{0.12} & \raisebox{-1.5ex}{0.00} & \raisebox{-1.5ex}{0.90} & \raisebox{-1.5ex}{0.00} & \raisebox{-1.5ex}{0.00} & \raisebox{-1.5ex}{0.26} & \raisebox{-1.5ex}{0.49} \\ 
 \hline
   Setting at GL 2 &  &  &  & FS &  & BS & & SH &  & Ov \\ 
  \hline
res=50m ts=15;21 tc=0.0002;0.002 &  &  &  & \raisebox{-1.5ex}{0.23} &  & \raisebox{-1.5ex}{0.87} & & \raisebox{-1.5ex}{0.38} &  & \raisebox{-1.5ex}{0.63} \\ 
  \hline
\end{tabular}
\label{table:fuzzy_meso}
\end{table}
\subsection{tpi}
macro: 
At GL 2, the search radii chosen are both larger, being 150 and 700 m, and the computations are based on the 50 m DTM. The addition of a map based on a setting with an even larger range between the search radii (50 m and 200 m) improved the classification accuracy from 51 to 52\% by increasing the number of data points correctly mapped to the surveyed position FS and slightly decreasing the accuracy of the the three remaining positions.

\begin{table}[!htbp]
\caption{Accuracy values of  TPI-based landform maps computed  with the best parameter setting for each topographic position at meso scale. "plus"  indicates that the parameter setting described is combined with the aforementioned parameter setting. res = DTM resolution, R1 = inner search radius, R2 = outer search radius, Ov = Overall accuracy}
\centering
\begin{tabular}{p{2.8cm}|rrrrrrrrrr}
  \hline
 & FL & LO & DA & FS & SF & BS & SS & SH & RI & OCR \\ 
  \hline
{res=10m R1=50m R2=90m} & \raisebox{-1.5ex}{0.30} & \raisebox{-1.5ex}{0.32} & \raisebox{-1.5ex}{0.00} & \raisebox{-1.5ex}{0.13} & \raisebox{-1.5ex}{0.00} & \raisebox{-1.5ex}{0.93} & \raisebox{-1.5ex}{0.00} & \raisebox{-1.5ex}{0.00} & \raisebox{-1.5ex}{0.30} & \raisebox{-1.5ex}{0.50} \\ 
 \hline
   Setting at GL 2 &  &  &  & FS &  & BS & & SH &  & OCR \\ 
  \hline
{res=10m R1=80m R2=300m} &  &  &  & \raisebox{-1.5ex}{0.39} &  & \raisebox{-1.5ex}{0.86} & & \raisebox{-1.5ex}{0.35} &  & \raisebox{-1.5ex}{0.63} \\ 
  \hline
\end{tabular}
\label{table:tpi_meso}
\end{table}

At the more generalized GL 2 with only three topographic positions, the accuracy is increased to 63\% based on the search radii 80 and 300 m. As is the case for GL 1, the introduction of a second parameter setting does not significantly improve the accuracy. 

\subsection{geoms}

 
Additionally to summarizing the performance at GL 1, Table \ref{table:geom} shows how the reduced dataset and legend at generalisation level 2 slightly improve overall accuracy when applying the same parameter setting. The addition of a further landform element map based on a lower flatness threshold as well as a larger search radius increases the correct classification rate of RI and FS at the cost of BS.

\begin{table}[!htbp]
\caption{Accuracy values of geomorphon-based landform maps computed  with the best parameter setting for each topographic position at macro scale. "plus"  indicates that the parameter setting described is combined with the aforementioned parameter setting. res = DTM resolution, L = search radius, fl = flatness threshold, Ov = Overall accuracy}
\centering
\begin{tabular}{p{2.8cm}|rrrrrrrr}
  \hline
setting at GL 1 & FL & LO & DA & FS &  BS  & SH & RI & Ov \\ 
  \hline
res=50m fl=10 L=400m & \raisebox{-1.5ex}{0.38} & \raisebox{-1.5ex}{0.49} & \raisebox{-1.5ex}{0.20} & \raisebox{-1.5ex}{0.00} & \raisebox{-1.5ex}{0.81} & \raisebox{-1.5ex}{0.00} & \raisebox{-1.5ex}{0.37} & \raisebox{-1.5ex}{0.49}  \\ 
 \hline
 setting at GL 2 & CH &  &  & FS &  BS  & SH &  & Ov \\ 
  \hline
res=50m fl=10 L=400m & \raisebox{-1.5ex}{0.50}  &  &  & \raisebox{-1.5ex}{0.13} &   \raisebox{-1.5ex}{0.81} & \raisebox{-1.5ex}{0.36} &  & \raisebox{-1.5ex}{0.52} \\ 
plus res=50m fl=1 L=1500m & \raisebox{-1.5ex}{0.50} &&& \raisebox{-1.5ex}{0.16} & \raisebox{-1.5ex}{0.80} & \raisebox{-1.5ex}{0.39}& & \raisebox{-1.5ex}{0.54} \\ 
  \hline
\end{tabular}
\label{table:geom_macro}
\end{table}
A reduction of data points and number of classes according to generalisation level 2 leads to a correct classification rate of 62\% applying a flatness threshold of 8 degrees and a search window of 150 m to the 10 m DTM. The classification rate is improved to 64\% by adding the landform map based on the 50m DTM, a search radius of 250 m and a flatness threshold of 10 degrees.

\begin{table}[!htbp]
\caption{Accuracy values of geomorphon-based landform maps computed  with the best parameter setting for each topographic position at meso scale. "plus"  indicates that the parameter setting described is combined with the aforementioned parameter setting. res = DTM resolution, L = search radius, fl = flatness threshold, Ov = Overall accuracy}
\centering
\begin{tabular}{p{2.8cm}|rrrrrrrrrr}
  \hline
 & FL & LO & DA & FS & SF & BS & SS & SH & RI & Ov \\ 
  \hline
{res=10m fl=8 L=80m} & \raisebox{-1.5ex}{0.00} & \raisebox{-1.5ex}{0.17} & \raisebox{-1.5ex}{0.00} & \raisebox{-1.5ex}{0.00} & \raisebox{-1.5ex}{0.00} & \raisebox{-1.5ex}{0.92} & \raisebox{-1.5ex}{0.00} & \raisebox{-1.5ex}{0.00} & \raisebox{-1.5ex}{0.38} & \raisebox{-1.5ex}{0.49} \\ 
 \hline
   Setting at GL 2 &  &  &  & FS &  & BS & & SH &  & Ov \\ 
  \hline
{res=10m fl=8 L=150m} &  &  &  & \raisebox{-1.5ex}{0.07} &  & \raisebox{-1.5ex}{0.90} & & \raisebox{-1.5ex}{0.39} &  & \raisebox{-1.5ex}{0.62} \\ 
plus res=50m fl=10 L=250m &  &  &  & \raisebox{-1.5ex}{0.35} &  & \raisebox{-1.5ex}{0.86} & & \raisebox{-1.5ex}{0.39} &  & \raisebox{-1.5ex}{0.64} \\
  \hline
\end{tabular}
\label{table:geom_meso}
\end{table}
\subsection{terrain}

At GL 2, the same set of predictors results in an overall accuracy of 59\%. Table \ref{table:terrain_macro} shows how additional terrain parameters improve the accuracy of the map based on the single best terrain parameter, topographic wetness index. Regarding the overall accuracy, the topographic position index with a search radius of 200 m based on the 10 m DTM performs similar to the topographic wetness index, however it fails to classify any data points as FS. 

 At GL 2, a topographic position index with a slightly larger search window of 90 m was selected as the single parameter best representing the 3 generalisations at meso scale.

\begin{table}[!htbp]
\caption{Correct classification rates for maps computed with regional and local terrain parameters for each topographic position at meso-scale and with a radial kernel SVM. res = DTM resolution,r=radius, ws=search window size, topographic position index, OCR = Overall correct classification rate}
\centering
\begin{tabular}{p{3cm}|rrrrrrrrrr}
  \hline
Setting at GL 1 & FL & LO & DA & FS & SF & BS & SS & SH & RI & OCR \\ 
  \hline
{TPI res=10m r=70m} & {0.00} & {0.31} &{0.00} & {0.00} & {0.00} & {0.94} & {0.00} & {0.05} & {0.35} & {0.51} \\ 
plus Slope res=50m ws=150m & \raisebox{-1.5ex}{0.45} & \raisebox{-1.5ex}{0.33} & \raisebox{-1.5ex}{0.25} & \raisebox{-1.5ex}{0.08} & \raisebox{-1.5ex}{0.00} & \raisebox{-1.5ex}{0.93} & \raisebox{-1.5ex}{0.00} & \raisebox{-1.5ex}{0.03} & \raisebox{-1.5ex}{0.35} & \raisebox{-1.5ex}{0.52} \\
 \hline
 Setting at GL 2 &  &  &  & FS &  & BS & & SH &  & OCR \\ 
 \hline
\raisebox{-0ex}{TPI res=10m r=90m} &  &  &  & \raisebox{-0ex}{0.14} &  & \raisebox{-0ex}{0.90} & & \raisebox{-0ex}{0.42} &  & \raisebox{-0ex}{0.64} \\ 
plus slope res=50m ws=150m &  &  &  & \raisebox{-1.5ex}{0.32} &  & \raisebox{-1.5ex}{0.91} & & \raisebox{-1.5ex}{0.46} &  & \raisebox{-1.5ex}{0.68} \\ 
  \hline
\end{tabular}
\label{table:terrain_meso}
\end{table}
\subsection{Comparison of the best parameter settings}
Table \ref{table:overall_comparison} gives an overview for how the different classification algorithms performed at different scales and generalisation levels.

\begin{table}[ht]
\caption{Cross-validated overall accuracy and (in brackets) kappa coefficient of the best representatives of each classification algorithm as well as the best performimg SVM-model based on a combination of single terrain parameters. }
\centering
\begin{tabular}{crrrr}
  \hline
Cross-validated overall accuracy & macro GL1 & macro GL2 & meso GL1 & meso GL2 \\ 
  \hline 
Dikau's curvature classification & 0.45 (0.16) & 0.47 (0.18) & 0.47 (0.12) & 0.60 (0.21) \\ 
  Wood's morphometric features & 0.46 (0.18) & 0.48 (0.20) & 0.47 (0.14) & 0.59 (0.21) \\ 
  Schmidt's fuzzy landform elements & 0.48 (0.25) & 0.53 (0.30) & 0.48 (0.19) & 0.63 (0.28) \\ 
  TPI-based landforms & 0.47 (0.21) & 0.51 (0.26) & 0.50 (0.20) & 0.51 (0.32) \\ 
  Geomorphon-based forms & 0.48 (0.25) & 0.52 (0.28) & 0.48 (0.15) & 0.62 (0.24) \\ 
  Terrain parameters & 0.51 (0.32) & 0.57 (0.38) & 0.52 (0.25) & 0.66 (0.38) \\ 
   \hline
\end{tabular}
\label{table:overall_comparison}
\end{table}

To compare the similarity of the best parameter settings of the different classification, the percentage of datapoints that were classified as the same topographic position, as well as Cohen's kappa, were calculated for each pair of algorithms (Table \ref{table:similarity_matrix}), generalisation levels and scale. 
\begin{table}[ht]
\caption{Similarity of the classification methods as described by the percentage of data points classified as the same topographic position at GL 1/GL 2.The values below the diagonal refer to meso scale, while those above represent macro scale. DCC = Dikau's curvature classification, WMF = Woods morphometric features, SFL = Schmidt's fuzzy landform elements, TBL = TPI-based landforms, GBF = geomorphon-based forms, TP = Terrain parameters}
\centering
\begin{tabular}{ccccccc}
  \hline
\%  & DCC & WMF &SFL &TBL & GBF & TP \\ 
  \hline
DCC &1 & 0.76/0.76 & 0.70/0.62 & 0.77/0.71 & 0.70/0.71 & 0.68 \\ 
WMF &0.85/0.77  & 1 & 0.68/0.62 & 0.74/0.68 & 0.74/0.74 & 0.69/0.65 \\ 
SFL & 0.83/0.90 & 0.89/0.75 & 1 & 0.70/0.64 & 0.67/0.63 & 0.70/0.66 \\ 
TBL & 0.83/0.74 &0.82/0.76  &0.81/0.74  & 1 & 0.74/0.69 & 0.71/0.71 \\ 
GBF &0.81/0.80  &0.82/0.81  & 0.80/0.79  & 0.82/0.77 & 1 & 0.66/0.66 \\ 
TP &0.79/0.79  &0.79/0.79  &0.80/0.83  &0.90/0.80  &0.82/0.83  & 1 \\ 
   \hline
\end{tabular}
\label{table:similarity_matrix}
\end{table}
\section{Discussion}

\clearpage
\section*{References}
\bibliography{p1.bib}

\end{document}